\documentclass{beamer}
\usepackage[utf8]{inputenc}
\usepackage{pgf,tikz}
\usetikzlibrary{arrows}
\usetikzlibrary{matrix,chains,positioning,decorations.pathreplacing,arrows}
\usepackage{xcolor}
\usepackage{times}
\usepackage{amsmath}
\usepackage{verbatim}
\usepackage{hyperref}
%\usepackage[colorlinks=true,urlcolor=blue]{hyperref}
\usepackage{pdfpages}
\usepackage{graphicx}
\usepackage{hyperref}

% \usepackage{float}
% \usepackage{caption}
% \usepackage{subcaption}

%\usetheme{CambridgeUS} %Good
\setbeamertemplate{itemize item}[ball]
\setbeamertemplate{itemize subitem}[ball]

%\usepackage{enumerate}
%\numberwithin{equation}{section}


\title{Multiple-response models for multivariate binary response}
\author{Steve Cygu \& Prof. Jonathan Dushoff\\ \vspace{1cm} McMaster University, Canada.}
\date{April 09, 2019}

\begin{document}
\frame{\titlepage}

\frame{\small{\frametitle{OUTLINE}\tableofcontents}} 


\section{Background}

\begin{frame}{Background}
\begin{itemize}[<+->]
\item Longitudinal (2003 - 2015) NUHDSS covering Korogocho and Viwandani
\item Predictors
\begin{itemize}[<+->]
\item Slum area
\item Interview year
\item Age
\item Gender
\item Ethnicity
\item Household size
\item Wealth index
\item Household expenditure
\end{itemize}
\item Response(s): Three WASH variables were created as per WHO definition

\begin{itemize}[<+->]
\item Drinking water source
\item Toilet facility type
\item Garbage disposal method
\end{itemize}
\end{itemize}

\end{frame}


\section{Objective}

\begin{frame}{Objective}
The aim is to investigate the contribution of demographic, socialland economic factors to improved waster, sanitation and hygien (WASH) among the urban poor.
\end{frame}

\begin{frame}{Problems}
How do we account for the repeated measurements within the households accross the years?
\begin{itemize}[<+->]
\item Model the wash varaibles seperately (Manuscript already submitted?)
\item Pick one of the WASH indicator and treat the remaining two as fixed covariates
\end{itemize}
\pause
\textcolor{red}{The two appraches are not accounting for the unmeasured variations and correlation among the WASH variables} 
\end{frame}

\begin{frame}{Problems}
Does this matter?

\begin{itemize}[<+->]
\item Aceess to any of the WASH indicators (variables) vary within the households and also accross the years.
\end{itemize}
\pause
\begin{columns}[t]
\column{.35\textwidth}
\includegraphics[width=\textwidth]{water_plot.pdf}
\pause
\column{.35\textwidth}
\includegraphics[width=\textwidth]{toilet_plot.pdf}
\pause
\column{.35\textwidth}
\includegraphics[width=\textwidth]{garbage_plot.pdf}
\end{columns}
\end{frame}

\begin{frame}{Problems}

There are several questions we can ask:
\begin{itemize}[<+->]
\item[1.] How do we pull information from (accross) the three levels (WASH)?
\begin{itemize}[<+->]
\item Can we account for household level variations?
\end{itemize}
\item[2.] Can we estimate the correlation between the WASH variables while controlling for the other covariates?
\item[3.] Can we control for the variation and still get a reliable estimate of the covariance?
\end{itemize}

\pause
What we may know?
\begin{itemize}[<+->]
\item \textcolor{red}{This is a difficult problem}!!!
\item \textcolor{gray}{Generalised linear mixed effects models may be a good starting point}
\item But we need some understanding of data generation process
\begin{itemize}[<+->]
\item \textcolor{blue}{Some simulations}
\end{itemize}
\end{itemize}

\end{frame}

\section{Methods}
\begin{frame}{Methods}
Assume that we only observed the predictors and simulate response
\begin{itemize}[<+->]
\item Pick one of the predictors - \textcolor{blue}{wealth index}, $x$
\item There are some unobserved confounders, $U$, effect
\item Assume we know the effect sizes, $\beta$s
\pause
\[
\hat{y}_i = U_i + \beta_{0i} + \beta_{1i}x
\]
Where $i\in\{\text{WASH}\}$
\item Let $P$ be the probability that HH has access to $i\in\{\text{WASH}\}$
\[
P = \frac{1}{1 + \exp{(-\hat{y_i})}}
\]
\item Simulate response, $y_i$ from a binomial distribution with probability of observing WASH $P$
\end{itemize}
\pause
\textcolor{blue}{Now that we know the observed $\beta$s, can we find a model which gives us back the $\beta$s having answered the 3 question above?}
\end{frame}

\subsection{Simulations}
\begin{frame}{Simulation results}
\centering
\includegraphics[scale=0.4]{prop_plot.pdf}
\end{frame}

\subsection{Models}



\end{document}
